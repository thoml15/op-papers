\documentclass{article}
\usepackage{graphicx}
\usepackage{wrapfig}
%\usepackage{inconsolata}
\usepackage{enumerate}
\usepackage{hyperref}
\usepackage{verbatim}
\usepackage[parfill]{parskip}
\usepackage[margin = 2.5cm]{geometry}

\usepackage[T1]{fontenc}


\begin{document}

\title{Exercising your curiosity about Containers\\ IN720 Virtualisation}
\date{}
\maketitle

\section*{Introduction}
Throughout this paper, we're going to take a tour of different kinds of virtualisation.  To begin, we'll look at ``Containers,'' a form of {\em lightweight} virtualisation where isolated applications all share the same operating system kernel.

Your lecturer for this paper does not arrive in Dunedin until Week Two of the semester.  By then, I want us to be able to have fun trying stuff out!  All things going well, we will have a new virtualisation environment that we can use to test out Linux Containers and the Docker system.

Until then\ldots

\section*{Preparing for a Conversation about Containers}

IT professionals should be able to explain things {\em succinctly and coherently} when asked.  So here's a question for you:
\begin{quote}
{\em I understand why you might want a virtual machine.  Maybe you sell web servers as a service, and you want to run lots of servers on one physical machine.  Or maybe you need to develop under several operating systems at once---you can avoid installing them on multiple machines, or avoid rebooting to access each OS.  But I hear there are these things called Containers, and people are even talking about Containerisation\ldots What's a Container for? What's the typical use case? Is there anybody I've heard of who's using them and what are they doing?}
\end{quote}

\begin{itemize}
\item To start getting a handle on Containers, check out my new favourite site {\tt sysadmincasts.com}.
Episode 24 explains LXC: \url{http://bit.ly/2v6ksnt}. (Read/watch/listen to all their other stuff too; it's pretty good.)

\item InfoWorld has a short, up-to-date article on Docker at \url{http://bit.ly/2sthXcF}.

\item You may also like to read freecodecamp's introduction to containers: 
\url{http://bit.ly/2tYeZyX}.
\end{itemize}

Come to Week Two's session ready for some class discussion on Containers, and what you can do with them.

\end{document}
