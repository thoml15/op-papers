\documentclass{article}
\usepackage{graphicx}
\usepackage{wrapfig}
%\usepackage{inconsolata}
\usepackage{enumerate}
\usepackage{hyperref}
\usepackage[margin = 2.25cm]{geometry}
\setlength\parindent{0pt}

\begin{document}

\begin{figure}
\includegraphics[width=30mm]{../../../resources/images/oplogo.png}
\end{figure}

\title{Course Directive\\IN722 Hardware\\Semester Two, 2017}
\date{}
\maketitle

\section*{Description}
Modern networked systems require a broad set of skills, ranging from fundamental principles of electronics, networking, software development, as well as being able to tackle coordination challenges to integrate individual components in a coherent system. 

The course introduces students to contemporary developments in the context of the `Internet of Things' and highlights future emerging fields. Topics include the construction and programming of embedded devices as well as coordination software that facilitates Machine-to-Machine (M2M) communication. Beyond the purely technical perspective the course discusses conceptual challenges, such as the notion of `context', system coordination on a global scale as well as raising sensitivity for implications on security and privacy. 

The students integrate these different perspectives in a comprehensive project that fosters the students' capability to systematically develop, secure and maintain a context-sensitive distributed system for a specific application context. 


\section*{Course Information}
\begin{itemize}
  \item 15 Credits
  \item Prerequisites: IN621 and instructor permission
\end{itemize}

\section*{Lecturers}
\begin{tabular}[width=7cm]{l r}

  % after \\: \hline or \cline{col1-col2} \cline{col3-col4} ...
  Nathan Rountree \\
     Office: & D316a \\
     Phone: & 972 7447 \\
     Email: & \texttt{nathan.rountree@op.ac.nz} \\
\end{tabular}

\section*{Course Dates}
\begin{tabular}{lr}
Term 1 (11 weeks) & 17 July -- 29 September \\
Mid semester break & 2 October -- 13 October \\
Term 2 (5 weeks) & 16 October -- 17 November \\
\end{tabular}


\section*{Resources}

Lab documents, slides, and other material is available on Github at https://github.com/ncrountree/op-papers.

\newpage 

\section*{Course Content and Schedule}
This schedule is tentative and subject to change based on needs of the class.

\renewcommand{\arraystretch}{1.5}
\begin{tabular}{|l|c|l|}
\hline
 Week & Week Start & \multicolumn{1}{c|}{Topics}             \\ \hline
 1    & 17 Jul     & Overview of the Internet of Things      \\ \hline
 2    & 24 Jul     & Devices                                 \\ \hline
 3    &  31 Jul     & Devices, Network Communication          \\ \hline
 4    & 7 Aug     & Network Communication, Protocol Stacks  \\ \hline
 5    & 14 Aug     & M2M Communication                       \\ \hline
 6    & 21 Aug     & IoT Infrastructure                      \\ \hline
 7    & 28 Aug     & Coordination Protocols                  \\ \hline
 8    &  4 Sep     & Security                                \\ \hline
 9    & 11 Sep     & Project Work                            \\ \hline
 10   & 18 Sep     & Project Review                          \\ \hline
 H1   & 25 Sep     & Reviewing Existing IoT Solutions                                 \\ \hline
 H2   &  2 Oct     & Holiday                                 \\ \hline
 11   & 9 Oct     &  Holiday       \\ \hline
 12   & 16 Oct     & Project Work (Infrastructure Revision)  \\ \hline
 13   & 23 Oct     & Project Work                            \\ \hline
 14   &  30 Oct     & Project Refinement (Functionality)      \\ \hline
 15   &  6 Nov     & Project Refinement (Security, Hardening)\\ \hline
 16   & 13 Nov     & Project Evaluation                      \\ \hline
\end{tabular}

\section*{Assessment}

%Assessments are weighted as follows: \\
\begin{tabular}{|l|c|}
\hline
Assessment                  &  Weighting \\ \hline
Project Work                &  100\% \\ \hline
\end{tabular}

\section*{Criteria for Passing}
You must receive an overall average mark of 50\% or higher to pass this paper.

\section*{Course Requirements and Expectations}
\subsection*{Attendance}
This paper is composed of a mix of lectures and self-paced project work.  Attendance is at your discretion. 
However, you are responsible for keeping up with events that take place in class and completing work on schedule. 

\subsection*{Communication}
Important announcements and discussions about the course, assessments, and scheduling may take place during class sessions.  It is your responsibility to be informed about them.  If you cannot attend a class session, be sure to check with another student.

%A private channel, \texttt{networks-admin}, is set up on the op-bit Slack at \url{https://op-bit.slack.com/}.  The channel is intended for general class discussion.  Important announcements may also be posted there, so you should join and monitor the channel.

Your student email is an official communication channel. It is your responsibility to regularly check your student email for important course related material, including changes to class scheduling or assessment details. Not checking will not be accepted as an excuse.

You can manage your email at the Student Hub and download the instructions for forwarding your email at http://www.op.ac.nz/students/student-hub/

\subsection*{Polytechnic Closure}
In the event that the Polytechnic is closed or has a delayed opening because of snow or bad weather, you should not attempt to attend class if it is unsafe to do so. It is possible that your instructor will not be able to attend either, so classes will not physically be meeting. However, this does not become a holiday. Rather, material will be available on the Cisco Academy web site covering the material for classes affected by the closure. You are responsible for any material presented in this manner. Information about closure will be posted on the Otago Polytechnic facebook page https://www.facebook.com/OtagoPoly.

\subsection*{Group Work and Originality}
Students in the Bachelor of Information Technology degree are expected to hand in original work.  Students are encouraged to discuss
assignments with their fellow students.  However, all assignments are to be completed as individual works unless group work is explicitly involved.
Failure to submit your own unique work will be treated as plagiarism.

\subsection*{Referencing}
Appropriate referencing is required for all work.  Referencing standards will be specified by your instructor.

\subsection*{Plagiarism}
Plagiarism is submitting someone else's work as your own.  Plagiarism offences are taken seriously and an
assessment that has been plagiarised may be awarded a zero mark.  A definition of plagiarism is in the Student Handbook,
available online or at the school office.

\subsection*{Submission Requirements}
All assignments are to be submitted by the time, date, and method given when the assignment is issued.

\subsection*{Extensions}
Extensions are only available for unusual circumstances.  These must be applied for, and approved, prior to the submission deadline.

\subsection*{Impairment}
In case of sickness contact your lecturer or year co-ordinator as soon as possible, preferably before the test or
assignment is due.  The policy regarding the granting of a mark that considers impaired performance requires a medical
certificate and a medical practitioners signature on a form. You should refer to the guide on impaired performance
in the student handbook.

\subsection*{Appeals}
If you are concerned about any aspect of your assessment, please approach the lecturer in the first instance.  We support
an open door policy and aim to resolve issues promptly.  Further support is available from the Programme
Manager and Head of School. Otago Polytechnic has a formal process for academic appeals if necessary.

\subsection*{Other Documents}
Regulatory documents relating this course can be found on the Polytechnic website.




\subsection*{Special Resources and Requirements}
If you have any special needs, whether they relate to the course material, the exercises, the assessment, or anything in the course -
then \textit{please} let your instructor know as soon as possible.

\end{document}
