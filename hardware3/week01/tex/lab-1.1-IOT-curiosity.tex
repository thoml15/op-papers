\documentclass{article}
\usepackage{graphicx}
\usepackage{wrapfig}
%\usepackage{inconsolata}
\usepackage{enumerate}
\usepackage{hyperref}
\usepackage{verbatim}
\usepackage[parfill]{parskip}
\usepackage[margin = 2.5cm]{geometry}

\usepackage[T1]{fontenc}


\begin{document}

\title{Exercise your curiosity about the Internet of Things\\ IN722 Hardware}
\date{}
\maketitle

\section*{Things to start thinking about in Week One}

Your lecturer for this paper won't be arriving in Dunedin until Week Two.
That doesn't mean you get a week off!   When I arrive, I want to have an
intelligent conversation with you about the Internet of Things.

This class is not that big, so everybody will get to contribute.

Think about the following things, and come prepared to talk about about them.  You don't
have to write an essay, but I would strongly  suggest that you:
\begin{itemize}
\item Read widely
\item Make some notes
\item Organise your thoughts
\item Draw some diagrams
\end{itemize}

Here's some starting points for our conversation:

\begin{itemize}
\item Arduino and Raspberry Pi are both products that allow you to embed considerable computing
power into quite small items at quite low cost.  Explain the differences between the two products.
Imagine a project that would be perfect for using Arduino, but not so good for Raspberry Pi.  How about the other way around?
\item Suppose you were intending to build a product that needed to report its location, using GPS
coordinates, to a remote server.  What collection of components (Arduino or Raspberry Pi
or indeed, anything else you think might be better) would enable this to happen?  How would they have to be configured?  What constraints would you be introducing with your components/configuration?
\end{itemize}

N.B. There are no right or wrong answers to these questions---only a creative space to be explored.
The more exploration you do, the more fun we'll have during class time.

\end{document}
